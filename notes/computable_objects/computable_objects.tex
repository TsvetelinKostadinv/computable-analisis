\section{Други изчислими обекти}
Вдъхновено от задача за домашно.

Нека имаме $\rho_S : \mathbb T \to \mathcal P(\N)$ - някакво представяне на подмножества от естествени числа.
\marginpar{Дали има значение какво е $\rho_S$?}

Тогава нека $t_L, t_R \in \mathbb T \mapsto t$ реализирано по следния начин:
\begin{equation}
    t(i) = \pi (t_L(i), t_R(i))
\end{equation}

\begin{definition}
    $\mathbb No$ е мн-вото на сюреалните числа \cite{knuth1974surreal}.
    \begin{equation}
        \N o = \set{x \mid x: \subseteq Ord \to \set{+, -}}
    \end{equation}
\end{definition}