\section{Представяне в основа $b \geq 2$}
От 2024-03-20
\begin{quote}
    Говорим за изчислимост, но дали наистина разликата $|f(n) - \alpha|$ наистина е ефективно изчислима? По-точно как ефективно изчисляваме разлика между рационално и реално число?
\end{quote}
Наистина, ако $\alpha$ е изчислимо, то самата разлика е изчислима, защото $f$ е изчислима по дефиниция.

Днес - граница на изчислима редица може да не е изчислимо число.

\subsection{Пример}
Нека разглеждаме $x^3 + 3x^2 + p$, където $p$ е параметър.

\subsection{Итерация}
\begin{notation}
    Нека $f: \R \to \R$ - изчислима
    \begin{equation*}
        \begin{split}
            g(0, \alpha) &= \alpha \\
            g(1, \alpha) &= f(\alpha) \\
            g(n+1, \alpha) &= f(g(n, \alpha))
        \end{split}
    \end{equation*}
\end{notation}
\begin{problem}
    Търсим $n$, т.ч $g(n, \alpha) = 0$
\end{problem}
\begin{solution}
    Това беше задача на Стефанов, не си спомням точно замисъла.
\end{solution}

\subsection{Представяне в основа $b \geq 2$}
Нека $M \geq 0$ и наричаме цифри $\left\{0, 1, \dots, \underbrace{b-1}_{\Bar{0}}\right\}$

\begin{notation}
    $\langle M. D_1, D_2, \dots, D_n\rangle_b = M + \sum\limits_{i=1}^{n}D_i.b^i$
\end{notation}
\begin{definition}
    Нека $\xi \in \R^{\geq 0}$, казваме, че $\langle M. D_1, D_2, \dots, D_n, \dots\rangle_b$ представя $\xi$ в основа $b$, ако за вс. $n \in \N$:
    \begin{equation*}
        \langle M. D_1, D_2, \dots, D_n\rangle_b \leq \xi \leq \langle M. D_1, D_2, \dots, D_n\rangle_b + b^{-n}
    \end{equation*}
\end{definition}
Първо отделяме цялата част и после търсенето се случва като разделяме интервала на $b$ равни части и избираме най-голямата цифра, която да бъде под търсеното $\xi$.

Ако стане така, че $\xi$ е на границата на такъв интервал, ще получим представяне от вида $M.D_1 D_2 \dots (D_n - 1) \Bar{0} \dots \Bar{0} \dots$.

\begin{proposition}
    Нека $s: \N \to \N$
    \begin{equation*}
        \begin{split}
            s(0) = M \\
            s(i) = D_i, i > 0
        \end{split}
    \end{equation*}
    Твърдим/Знаем, че
    \begin{equation*}
        \xi = \sum\limits_{i=0}^{\infty} \frac{s(i)}{b^i}
    \end{equation*}
    И можем да твърдим, че:
    \begin{enumerate}
        \item Всяка ф-ия $s: \N \to \N$ т.ч. $\forall i \geq 1: s(i) \leq b$ представя единствено $\xi \in \R^{\geq 0}$
        \item Ако $\xi \neq \frac{m}{b^k}$ или $\xi = 0$, то $\xi$ има единствено представяне в основа $b$
        \item Ако $\xi = \frac{m}{b^k}$ и $k$ е $min$, то $\xi$ има двойствено представяне в основа $b$
        \begin{equation*}
            \begin{split}
                M.D_1 D_2 \dots D_k 00\dots 0\dots\\
                M.D_1 D_2 \dots (D_k - 1) \Bar{0} \Bar{0} \dots \Bar{0} \dots
            \end{split}
        \end{equation*}
    \end{enumerate}
\end{proposition}
\begin{example}
    В десетично представяне на числата, ф-ията $\xi \mapsto 3\xi$ не е изчислима. Проблема е точно с числа, където $b$ е взаимно просто с числото, с което умножаваме.
\end{example}
\begin{proposition}
    $\xi \R^{\geq 0}$ е изчислимо $\iff$ има изчислимо представяне в основа $b$
\end{proposition}
\begin{proof}
    \begin{itemize}
        \item[$(\Leftarrow)$] Нека $M.D_1 D_2 \dots D_n \dots$ е изчислимо представяне на $\xi$ в основа $b$, т.е $s$ е изчислима
        
        Нека
        \begin{equation*}
            f(n) = \sum\limits_{i=0}^{n+1} \frac{s(i)}{b^i}
        \end{equation*}
        Тогава 
        \begin{equation*}
            |f(n) - \xi| = \sum\limits_{i=n+2}^\infty \frac{s(i)}{b^i} \leq \sum\limits_{i=n+2}^\infty \frac{b-1}{b^i} = \frac{1}{b^n} < \frac{1}{n+1}
        \end{equation*}

        \item[$(\Rightarrow)$] 
        \begin{enumerate}
            \item $\xi = \frac{m}{b^k}$ тогава очевидно $\xi$ има изчислимо представяне
            \item $\xi \neq \frac{m}{b^k}$. Имаме ф-ия $f: \N \to \Q$ т.ч.
            \begin{equation*}
                |f(n) - \xi| < \frac{1}{n+1} 
            \end{equation*}
            И разглеждаме:
            \begin{equation*}
                \begin{split}
                    0 < \xi - \sum\limits_{i=0}^{n} \frac{s(i)}{b^i} < \frac{1}{b^n} \\
                    b^n \xi = b^n \sum\limits_{i=0}^n \frac{s(i)}{b^i} = b.b^{n-1}\left(\sum\limits_{i=0}^{n-1} \frac{s(i)}{b^i} + \frac{s(n)}{b^n}\right) = b.\lfloor b^{n-1}\xi\rfloor + s(n)
                \end{split}
            \end{equation*}
            Значи трябва да покажем, че
            \begin{equation*}
                s(n) = \lfloor b^n \xi\rfloor - b.\lfloor b^{n-1}\xi\rfloor
            \end{equation*}
            наистина е изчислима функция. За това ни е необходимо $\lfloor b^n\xi\rfloor$ да е изчислимо.

            Имаме следното:
            \begin{equation*}
                \begin{split}
                    f(k) - \frac{1}{k+1} < \xi < f(k) + \frac{1}{k+1}\\
                    b^k(f(k) - \frac{1}{k+1}) < b^n\xi < b^k(f(k) + \frac{1}{k+1}) \\
                    \forall n, \exists k:\; m < b^k(f(k) - \frac{1}{k+1}) < b^n\xi < b^k(f(k) + \frac{1}{k+1}) < m+1 \\
                    m = \left\lfloor b^n \left(f(k) - \frac{1}{k+1}\right)\right\rfloor = \left\lfloor b^n \left(f(k) + \frac{1}{k+1}\right)\right\rfloor\\
                    g(n) = \mu k \left[ \left\lfloor b^n \left(f(k) + \frac{1}{k+1}\right)\right\rfloor < b^n \left(f(k) - \frac{1}{k+1}\right) \right]\\
                    \left\lfloor b^n \xi \right\rfloor = \left\lfloor b^n\left(f(g(n)) + \frac{1}{g(n) + 1}\right)\right\rfloor
                \end{split}
            \end{equation*}
            Значи е изчислимо
        \end{enumerate}
    \end{itemize}
\end{proof}

\section{Редици на Шпекер (Specker)}
Хехе, редици на Шмекер

Нека $A \subseteq \N$ и $x_A \in \R$ е реално число с двоично представяне $\langle D_0.D_1 D_2 \dots D_n \dots \rangle_2$, т.ч.
\begin{equation*}
    D_i = \begin{cases}
        1, i \in A \\
        0, i \notin A
    \end{cases}
\end{equation*}
\begin{proposition}
    $x_A$ е изчислимо $\iff$ $A$ е разрешимо
\end{proposition}
\begin{proof}
    Разбиваме на случаи
    \begin{itemize}
        \item[$(x_A = \frac{m}{2^k})$] $A$ е крайно или $\N - A$ е крайно. Очевидно $A$ е разрешимо
        \item[$(x_A \neq \frac{m}{2^k})$] $x_A$ е изчислимо $\iff$ представянето в основа 2 на $x_A$ е изчислимо $\iff$ характеристичната ф-ия на $A$ е изчислима
    \end{itemize}
\end{proof}

\begin{proposition}
    Нека $A \neq \emptyset$ - полу-разрешимо множество и неразрешимо. Тогава $A = range(f)$, където $f$ е инективна и изчислима $f: \N \to \N$
\end{proposition}
\begin{proof}
    \begin{equation*}
        A = \left\{ x \in \N \mid \exists y:\ h(x, y) = 0 \right\}
    \end{equation*}
    където $h$ е примитивно рекурсивна. Фиксираме $a \in A$
    \begin{equation*}
        g(n) = \begin{cases}
            L(n)& , h(L(n), R(n)) = 0 \\
            a& , h(L(n), R(n)) \neq 0
        \end{cases}
    \end{equation*}
    Нека 
    \begin{equation*}
        \begin{split}
            f(0) = g(0)
            f(n+1) = g(k)\text{, т.ч. } k = min g(k) \notin \left\{f(0), f(1), \dots, f(n)\right\}
        \end{split}
    \end{equation*}
    Тогава $f$ ще е инективна.
    \begin{equation*}
        x_A = \sum\limits_{i=0}^\infty \frac{1}{2^{f(i)}}
    \end{equation*}
    $x_A$ не е изчислимо, защото $range(f) = A$ не е разрешимо.

    От друга страна, ако разглеждаме:
    \begin{equation*}
        h(n) = \sum\limits_{i=0}^n \frac{1}{2^{f(i)}}
    \end{equation*}
    е изчислима и в някакъв смисъл $h(n) \underset{n \to \infty}{\to} x_A$. Така че редица от изчислими числа не задължително клони към изчислимо число.
\end{proof}
\begin{remark}
    Произволните редици не ни дават достатъчно гаранции. Ако разглеждаме строго монотонните редици, нещата стоят по различен начин.
\end{remark}