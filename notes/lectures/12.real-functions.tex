\section{Реални функции}
\begin{definition}
    $\theta: \subseteq \R^{N} \to \R$ - реална функция
\end{definition}
Искаме да именуваме $\R^N$ по някакъв начин. Ще работим с:
\begin{equation}
    d_N(\overrightarrow{\xi}, \overrightarrow{\eta}) = \max\{|\xi_1 - \eta_1|, \dots, |\xi_N - \eta_N|\} = \max\limits{i \leq N} \{|\xi_i - \eta_i|\}
\end{equation}
метрика в $\R^N$ (важи неравенството на триъгълника).

Лесно се съобразява, че $\Q^N$ са гъсто подмножество на $\R^N$. Ще дефинираме нотация на множеството $\alpha_N: \N \to \Q^N$:
\begin{equation}
    \begin{split}
        N > 1:\; \alpha_N(n) = (\nu_\Q((n)_0), \nu_\Q((n)_1), \dots, \nu_\Q((n)_{N-1}))
        N=1: \alpha_1 = \nu_\Q
    \end{split}
\end{equation}
И сега искаме именуваща система $\nu_N: \subseteq \mathbb T \to \R^N$:
\begin{equation}
    \nu_N(t) = \overrightarrow{\xi} \bydef d_N(\alpha_N(t(i)), \overrightarrow{\xi}) < \frac{1}{i+1}
\end{equation}
\begin{definition}
    $\overrightarrow{\xi}$ е $\nu_\N$-изчислим $\bydef$ съществува изчислимо $t: \N \to \N:\; \overrightarrow\xi = \nu_N(t)$
\end{definition}
Потвърждава досегашното ни разбиране:
\begin{equation}
    \begin{split}
        \text{съществува изчислимо }t: \N \to \N:\; \overrightarrow\xi = \nu_N(t) \\
        \iff \text{съществува изчислимо }t: \N \to \N,\ \forall i\in \N:\; d_N(|\alpha_N(t(i)), \overrightarrow\xi|) < \frac{1}{i+1} \\
        \iff \text{съществува изчислимо }t: \N \to \N,\ \forall i,\ k\in \N:\;|(\alpha_N(t(i)))_k, \xi_k| < \frac{1}{i+1} \\
        \iff \forall k:\; \xi_k \text{ е изчислимо реално число}
        {} \\
        t_1, \dots, t_N \text{ са изчислими имена на } \xi_1, \dots, \xi_N \\
        t(i) = \pi(t_1(i), \dots, t_N(i))
    \end{split}
\end{equation}
\begin{definition}
    $\theta: \subseteq \R^N \to \R$ е изчислимо, ако е $\langle \nu_N, \nu_1 \rangle$-изчислима, т.е. има изчислима $\langle \nu_N, \nu_1 \rangle$-реализация
\end{definition}
\begin{proposition}
    $\theta: \subseteq \R^N \to \R$ е изчислима $\iff$ съществуват $\underbrace{F, G, H}_{\text{изчислителна система}}$ - изчислими оператори, т.ч.:
    \begin{equation}
        \begin{split}
            \forall \overrightarrow\xi \in \dom\theta,\ \forall f_1, \dots f_N,\ g_1, \dots, g_N,\ h_1, \dots, h_N: \N \to \N,\ \forall k,\ i:\; \left[\left| \frac{f_k(i) - g_k(i)}{h_k(i) + 1}\right| < \frac{1}{i+1} \right]\\
            \longrightarrow \forall i:\; \left| \frac{F(\overrightarrow{f}, \overrightarrow{g}, \overrightarrow{h})(i) - G(\overrightarrow{f}, \overrightarrow{g}, \overrightarrow{h})(i)}{H(\overrightarrow{f}, \overrightarrow{g}, \overrightarrow{h})(i)} - \theta(\overrightarrow\xi)\right| < \frac{1}{i+1}
        \end{split}
    \end{equation}
\end{proposition}
\begin{proof}
    \begin{itemize}
        \item[$(\Rightarrow)$] Нека $\Gamma$ е изчислима $\langle \nu_N, \nu_1\rangle$-реализация на $\theta$.
        \begin{equation}
            \theta(\underbrace{\nu_N(t)}_{=\overrightarrow\xi}) = \nu_1(\Gamma(t))
        \end{equation}
        
        Нека е изпълнена предпоставката.
        
        Можем да дефинираме $F, G, H$ по следния начин:
        \begin{equation}
            \begin{split}
                \Delta(f_1, \dots, h_N)(i) = \left(\frac{f_1(i) - g_1(i)}{h_1(i) + 1}, \dots, \frac{f_N(i) - g_N(i)}{h_N(i) + 1}\right)\\
                F(f_1, f_2, \dots, h_N)(i) = (\Gamma (\Delta(f_1, \dots, h_N))(i))_0 \\
                G(f_1, f_2, \dots, h_N)(i) = (\Gamma (\Delta(f_1, \dots, h_N))(i))_1 \\ 
                H(f_1, f_2, \dots, h_N)(i) = (\Gamma (\Delta(f_1, \dots, h_N))(i))_2
            \end{split}
        \end{equation}
        \item[$(\Rightarrow)$] Нека $F, G, H$ - оператори удовлетворяващи условието. Търсим $\langle \nu_N, \nu_1 \rangle$-реализация $\Gamma$ на $\theta$. $t$ е $\nu_N$-име на $\overrightarrow\xi \in \dom\theta$
        \begin{equation}
            \Gamma(t)(i) = \pi(F(f_1, \dots, h_N)(i), G(f_1, \dots h_N)(i), H(f_1, \dots h_N)(i))
        \end{equation}
        За подходящи функции:
        \begin{equation}
            \begin{split}
                f_k(i) = ((t(i))_k)_0 \\
                g_k(i) = ((t(i))_k)_1 \\
                h_k(i) = ((t(i))_k)_2 \\
            \end{split}
        \end{equation}
    \end{itemize}
\end{proof}

\subsection{Интересно представяне}
\begin{itemize}
    \item С факториелите - \href{https://en.wikipedia.org/wiki/Factorial_number_system}{Factorial number system}
    \begin{equation}
        \rho_!(t) = \alpha \iff \sum_{i\in\N^+} t(i).i!
    \end{equation}
    \item \href{https://en.wikipedia.org/wiki/Dedekind_cut}{Dedekind cuts}
\end{itemize}

