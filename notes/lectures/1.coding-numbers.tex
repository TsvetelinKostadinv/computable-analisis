\section{Рационални числа}
Може да се направи директно без много коментари, но е хубаво да се види интуицията.

Нека кодираме $\Q$:
\begin{equation*}
    \langle m, n, p \rangle \mapsto \frac{m - n}{p + 1}
\end{equation*}
\begin{definition}
    $f: \N^k \to \Q$ е изчислима(примит. рекурсивна) $\iff$ ако съществуват $f_{1,2,3}: \N^k \to \N$ изчислими(примит. рекурсивни), т.ч.
    \begin{equation*}
        \begin{split}
            f(\textbf{n}) = \frac{f_1(\textbf{n}) - f_2(\textbf{n})}{f_3(\textbf{n}) + 1}
        \end{split}
    \end{equation*}
\end{definition}
\begin{example}
    $f: \N^k \to \Q: f(\textbf{n}) = q = const$
\end{example}
\begin{example}
    $f: \N^k \to \N$ - старата изчислимост
\end{example}
\begin{problem}
    Да проверим затвореност относно затвореност относно аритметичните операции.
    
    Нека $f,g :\N^k \to Q$ са изчислими. То $f + g, f-g, f.g, \forall n\left(g(\textbf{n}) \neq 0 \rightarrow \frac{f(\textbf{n})}{g(\textbf{n})} \right)$ са изчислими
\end{problem}
\begin{solution}
    Нека
    \begin{equation*}
    \begin{split}
        f = \frac{f_1 - f_2}{f_3 + 1} \\
        g = \frac{g_1 - g_2}{g_3 + 1}
    \end{split}
    \end{equation*}
    тогава
    \begin{equation*}
        \begin{split}
            (f+g)_1 = f_1(g_3 + 1) + g_1(f_3 + 1)\\
            (f+g)_2 = f_2(g_3 + 1) + g_2(f_3 + 1)\\
            (f+g)_3 = f_3.g_3 + f_3 + g_3
        \end{split}
    \end{equation*}
    Останалите операции се получават по аналогичен начин
\end{solution}

\section{Реални числа}
\begin{definition}
    Реалното число $\alpha$ е изчислимо(примитивно рекурсивно) $\iff$ съществува изчислима(примитивно рекурсивна) $f: \N \to \Q$, т.ч.
    \begin{equation}\label{eq:real-computable-def}
        \forall n\; |f(n) - \alpha| < \frac{1}{n+1}
    \end{equation}
\end{definition}
\begin{notation}
    $f$ се нарича \textbf{име} на $\alpha$
\end{notation}
\begin{remark}
    В дясната част на \eqref{eq:real-computable-def} можем да си изберем произволна изчислима (примитивно рекурсивна) ф-ия, която "клони" към 0. \textbf{Но} имаме проблем, че нямаме дефиниция на граница и ще трябва да въведем формализъм за "ефективна граница" на изчислими функции.
\end{remark}
\begin{fact}
    Всяко рационално число е изчислимо.
\end{fact}

\subsection{Полето на изчислимите числа}

\begin{lemma}\label{lem:sum-computable}
    Ако $\alpha, \beta$ са изчислими, то $\alpha + \beta$ е изчислимо
\end{lemma}
\begin{proof}
    Имаме $f, g$ - изчислими за $\alpha, \beta$ съответно, които удовлетворяват \eqref{eq:real-computable-def} (техните имена). И нека:
    \begin{equation*}
        h'(n) = f(n) + g(n)
    \end{equation*}
    \begin{equation*}
        \begin{split}
            |h'(n) - \alpha - \beta| \leq |f(n) - \alpha| + |g(n) - \beta| < \frac{2}{n+1}
        \end{split}
    \end{equation*}
    Имаме константа 2, която ни пречи за дефиницията. Значи $h(n) = h'(2n+1)$ ще ни свърши работа.
\end{proof}
\begin{lemma}\label{lem:neg-computable}
    Ако $\alpha$ е изчислимо, то $- \alpha$ е изчислимо
\end{lemma}
\begin{proof}
    Нека $f$ е име за $\alpha$, тогава $-f$ е име за $-\alpha$
\end{proof}
\begin{lemma}\label{lem:mul-computable}
    Ако $\alpha, \beta$ са изчислими, то $\alpha . \beta$ е изчислимо
\end{lemma}
\begin{proof}
    Нека $f, g$ са имена за $\alpha, \beta$ съответно. И нека:
    \begin{equation*}
        h(n) = f(n).g(n)
    \end{equation*}
    И оценяваме
    \begin{equation*}
        |h(n) - \alpha\beta| = |f(n).g(n) - g(n)\alpha + g(n)\alpha - \alpha\beta| \leq |g(n)|.|f(n) - \alpha| + |\alpha|.|g(n) - \beta|
    \end{equation*}
    Избираме $A, B\in \N$, т.ч. $|\alpha| < A, |g(n)| < B, \forall n$. За $g$ можем да го изискваме, защото $\left\{g(n)\right\}_{n\in\N}$ е сходяща редица. 
    
    И значи
    \begin{equation*}
        |h(n) - \alpha\beta| < \frac{A+B}{n+1}
    \end{equation*}
    И отместването, което ни върши работа $h(n) = h'((A+B)(n+1) \dot{-} 1)$
\end{proof}
\begin{lemma}\label{lem:div-computable}
    Ако $\alpha$ са изчислими, то $\frac{1}{\alpha}$ е изчислимо
\end{lemma}
\begin{proof}
    Нека $f$ е име за $\alpha$.
    \begin{equation*}
        h(n) = \frac{1}{f(n)}
    \end{equation*}
    И разглеждаме:
    \begin{equation*}
        |h(n) - \frac{1}{\alpha}| = \frac{|f(n) - \alpha|}{|f(n)|.|\alpha|}
    \end{equation*}
    Нека фиксираме 
    \begin{equation*}
        A, C \in \N^+:\; \forall n > C:\ |f(n)| \geq \frac{1}{A}
    \end{equation*}
    И от тук значи $|\alpha| > \frac{1}{A}$. Строго, защото \textbf{първо} избираме $A$ после $C$.
    \begin{equation*}
        \frac{|f(n) - \alpha|}{|f(n)|.|\alpha|} < \frac{A^2}{n+1},\; n \geq C
    \end{equation*}
    Дефинираме 
    \begin{equation*}
        h'(n) = \frac{1}{f\left((C+A^2)(n+1)\right)}
    \end{equation*}
    И грешката на $h'$ е
    \begin{equation*}
        |h'(n) - \alpha| \leq \frac{A^2}{(C+A^2)(n+1)} \leq \frac{1}{n+1}
    \end{equation*}
\end{proof}

\begin{corollary}
    От \lemref{lem:sum-computable}, \lemref{lem:neg-computable}, \lemref{lem:mul-computable}, \lemref{lem:div-computable} следва, че изчислимите реални числа образуват поле.
\end{corollary}
\begin{remark}
    При доказателство, че грешката между името и реалното число има определена форма ($\frac{1}{n+1}$) може да се случи, че в числител имаме константа или сублинейна функция ($\frac{sublinear}{n+1}$), която така или иначе клони към 0. Такава грешка може да се коригира с (линейно) отместване на аргумента на приближаващата ф-ия. 
\end{remark}
