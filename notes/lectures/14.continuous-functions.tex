\section{Изчислимост и непрекъснатост}
\begin{theorem}
    Всяка изчислима реална функция $\theta : D \to \R,\ D \subseteq \R^N$ е непрекъсната в $D$.  
\end{theorem}
\begin{proof}
    Интуитивно: разполагаме името на аргумента на някаква безкрайна лента. Имаме $F, G, H$ - изчислима реализация. Нека за първите $k$ клетки от изхода сме използвали $m$ клетки от входа. При предположението, че $F, G, H$ не могат да се "връщат" , за да пишат в предишни клетки. Тогава ако променим входната лента след тези $k$ клетки, без да променяме тези $m$ клетки от изхода.

    \begin{itemize}
        \item[($N=1$)] Фискираме $\varepsilon > 0$, избираме $a \in \N$, т.ч. $\frac{2}{a+1} \leq \varepsilon$. Нека $\xi \in D$ и фиксираме произволно име на $\xi$ - $\langle f, g, h\rangle$. Фиксираме реализация на $\theta$ - $\langle F, G, H \rangle$. И имаме, че:
        \begin{equation}
            \langle F(f,g,h), G(f,g,h), H(f,g,h) \rangle \text{ е име на } \theta(\xi)
        \end{equation}
        И нека:
        \begin{equation}
            \begin{split}
                F(f,g,h) = p \\
                G(f,g,h) = q \\
                H(f,g,h) = r
            \end{split}
        \end{equation}
        И важи, че:
        \begin{equation}
            \abs{\frac{p+q}{r+1} - \theta(\xi)} < \frac{1}{a+1}
        \end{equation}
        От компактността на изчислението, можем да изберем крайни функции $\theta_f \subseteq f,\ \theta_g \subseteq g,\ \theta_h \subseteq h$, т.ч.:
        \begin{equation}
            \begin{split}
                F(\theta_f, \theta_g, \theta_h) = p \\
                G(\theta_f, \theta_g, \theta_h) = q \\
                H(\theta_f, \theta_g, \theta_h) = r
            \end{split}
        \end{equation}
        $\langle f, g, h\rangle$ е име на $\xi$:
        \begin{equation}
            \forall t \in \N:\; \abs{\frac{f(t) - g(t)}{h(t) + 1} - \xi} < \frac{1}{t+1}
        \end{equation}
        Нека $E = \dom(\theta_f) \cup \dom(\theta_g) \cup \dom(\theta_g)$. Отбелязваме, че $E$ е крайно.

        Можем да изберем $\delta > 0$, т.ч.
        \begin{equation}\label{eq:delta-choice}
            \forall t \in E:\; \abs{\frac{f(t) - g(t)}{h(t) + 1} - \xi} + \delta \leq \frac{1}{t+1}
        \end{equation}
        $\delta$ зависи от $f, g, h$.

        Нека $\xi' \in D,\ |\xi - \xi'| < \delta$.
        \begin{equation}
            \forall t \in E:\; \abs{\frac{f(t) - g(t)}{h(t) + 1} - \xi'} \leq \abs{\frac{f(t) - g(t)}{h(t) + 1} - \xi} + \abs{\xi - \xi'} \overset{\eqref{eq:delta-choice}}{<} \frac{1}{t+1}
        \end{equation}
        Избираме $\langle f', g', h' \rangle$ име на $\xi'$, т.ч. в/у $E$ съвпадат с $\langle f, g, h \rangle$: $\forall t \in E:\; f'(t) = f(t) = \theta_f(t), g'(t) = g(t) = \theta_g(t), h'(t) = h(t) = \theta_h(t)$.

        Разглеждаме:
        \begin{equation}
            \langle F(f', g', h'), G(f', g', h'), H(f', g', h') \rangle \text{ е име на } \theta(\xi')
        \end{equation}
        Но имаме, че:
        \begin{equation}
            \begin{split}
                F(f',g',h') = p \\
                G(f',g',h') = q \\
                H(f',g',h') = r
            \end{split}
        \end{equation}
        И имаме, че:
        \begin{equation}
            \abs{\frac{p-q}{r+1} - \theta(\xi')} < \frac{1}{a+1}
        \end{equation}
        Тогава:
        \begin{equation}
            \abs{\theta(\xi) - \theta(\xi')} \leq \abs{\theta(\xi) - \frac{p+q}{r+1}} + \abs{\frac{p-q}{r+1} - \theta(\xi')} < \frac{1}{a+1} + \frac{1}{a+1} \leq \varepsilon
        \end{equation}
    \end{itemize}
\end{proof}

\section{Ефективна равномерна непрекъснатост}
\begin{lemma}[Частична разрешимост]
    За всеки $\mu$-рекурсивен функционал $F(f, \overrightarrow{x})$, където $F \subseteq \mathcal F_1 \times \N^k \to \N$, съществува $u_F : \subseteq \N^{k+1} \to \N$, т.ч.:
    \begin{equation}
        \begin{split}
        \forall f: \N \to \N, \overrightarrow{x} \in \N^k: \\
        c = \langle f(0), f(1), \dots, f(len(c)) \rangle \\
        u_F(c, \overrightarrow{x}) = \begin{cases}
            r+1 &, F(f, x) \downarrow = r \& \text{ в това изчисление се изолзват $f(i)$ за } i \leq len(c)\\
            0 &, F(f, \overrightarrow{x}) \text{ използва } f(i),\ i > len(c) \\
            \uparrow &, F(f, \overrightarrow{x})\uparrow 
        \end{cases}
        \end{split}
    \end{equation}
\end{lemma}
\begin{proof}
    Индукция по $F$
    \begin{itemize}
        \item[(База 1)] Нека $F$ не зависи от $f$:
        \begin{equation}
            F(f, \overrightarrow{x}) = p(\overrightarrow{x})
        \end{equation}
        За $p$ - тотална и изчислима. Тогава:
        \begin{equation}
            u_F(c, \overrightarrow{x}) = p(\overrightarrow{x}) + 1
        \end{equation}
        \item[(База 2)] 
        \begin{equation}
            F(f, x) = f(x)
        \end{equation}
        Тогава:
        \begin{equation}
            u_F(x) = \begin{cases}
                (c)_x + 1 &, x \leq len(c) \\
                0 &, x > len(c)
            \end{cases}
        \end{equation}
        \item[(Стъпка)] Суперпозиция
        \begin{equation}
            F(f, \overrightarrow{x}) = F_0(f, F_1(f, \overrightarrow{x}), F_2(f, \overrightarrow{x}), \dots, F_m(f, \overrightarrow{x}))
        \end{equation}
        Конструираме:
        \begin{equation}
            \begin{split}
                u_F(c, \overrightarrow{x}) = v(c, u_{F_1}(c, \overrightarrow{x}), \dots, u_{F_m}(c, \overrightarrow{x})) \\
                u_{F_0}, \dots u_{F_m} \text{ от индукционната хипотеза} \\
                v(c, s_1, \dots, s_m) = \begin{cases}
                    u_{F_0}(c, s_1 \dot - 1, \dots, s_m \dot - 1) &, s_1\dots s_m \neq 0 \\
                    0 &, s_1 \dots s_m = 0
                \end{cases}
            \end{split}
        \end{equation}
        \item[(Стъпка)] Примитивна рекурсия
        \begin{equation}
            \begin{split}
                F(f, \overrightarrow{x}, 0) = F_0(f, \overrightarrow{x}) \\
                F(f, \overrightarrow{x}, t+1) = F_1(f, \overrightarrow{x}, t, F(f, \overrightarrow{x}, t))
            \end{split}
        \end{equation}
        Конструираме:
        \begin{equation}
            \begin{split}
                u_F(c, \overrightarrow{x}, 0) = u_{F_0}(c, \overrightarrow{x}) \\
                u_F(c, \overrightarrow{x}, t+1) = \begin{cases}
                    u_{F_1}(c, \overrightarrow{x}, t, u_F(c, \overrightarrow{x}, t) \dot - 1) &, u_F(c, \overrightarrow{x}, t)\neq 0 \\
                    0 &, u_F(c, \overrightarrow{x}, t) = 0 
                \end{cases}
            \end{split}
        \end{equation}
        \item[Стъпка] Минимизация
        \begin{equation}
            F(f, \overrightarrow{x}) = \mu\ t:\; F_0(f, \overrightarrow{x}, t) = 0
        \end{equation}
        Нека:
        \begin{equation}
            \begin{split}
                v(c, \overrightarrow{x}) = \mu\ t:\; u_{F_0}(c, \overrightarrow{x}, t) \leq 1 \\
                u_F(c, \overrightarrow{x}) = \begin{cases}
                    0 &, u_{F_0}(c, \overrightarrow{x}, v(c, \overrightarrow{x})) = 0 \\
                    v(c, \overrightarrow{x}) + 1 &, \text{иначе}
                \end{cases}
            \end{split}
        \end{equation}
    \end{itemize}
\end{proof}
\begin{corollary}
    Ако $F(f, \overrightarrow{x})$ е дефинирано за всяка тотална $f$ и $x \in \N^k$, то $u_F$ е тотална.
\end{corollary}

\begin{lemma}[равномерност]\label{lem:uniformity}
    Нека $\Gamma : \mathcal F_1 \to \mathcal F_1$ е $\mu$-рекурсивен оператор, т.ч. $\Gamma(f)$ е тотална за всяка тотална $f$.

    За всяка изчислима $g: \N \to \N$, съществува $\tilde g : \N \to \N$, което $\forall f$ - тотална, която се мажорира от $g,\ \forall a \in \N:\; \Gamma(f)(a) = \Gamma(f \upharpoonright \set{0, 1, \dots, \tilde g(a)})(a)$
\end{lemma}
\begin{proof}
    \begin{itemize}
        \item[($\exists$)] Ще доказваме:
        \begin{equation}
            \exists m\ \forall \underbrace{\varphi : \set{0, 1, \dots, m} \to \N}_\text{което се мажорира от $g$}: a \in \dom(\Gamma(\varphi))
        \end{equation}
        Допускаме противното:
        \begin{equation}
            \forall m\ \exists \underbrace{\varphi : \set{0, 1, \dots, m} \to \N}_\text{което се мажорира от $g$}: a \notin \dom(\Gamma(\varphi))
        \end{equation}
        Построяваме дърво $T$ от всички такива ф-ии $\varphi$. Ще дефинираме:
        \begin{equation}
            \psi \text{ е дете на } \varphi \iff \varphi \subset \psi \&  \exists a \in \N:\; \dom \psi = \dom\varphi \cup \set{a}
        \end{equation}
        Коренът е $\emptyset$. От допускането $T$ е безкрайно дърво. Защо всеки връх има крайна разклоненост?

        Съществено е, че ф-иите се мажорират от $g$.
        \begin{equation}
            \forall \varphi \in T,\ i \in \dom\varphi:\; \varphi(i) \in \set{0, 1, \dots g(i)}
        \end{equation}

        Лемата на Кьониг важи. значи има безкраен път $f: \N \to \N$, т.ч. $f \upharpoonright \set{0, 1, \dots m} \in T,\; \forall m$. Значи $\Gamma(f)(a)$ е дефинирано.

        От компактността следва, че $a \in \dom \Gamma(f \upharpoonright \set{0, 1, \dots m} )$ за някое $m$ - абсурд. Противоречи с построението на $f \upharpoonright \set{0, 1, \dots m} \in T$
        \item[(Ефективност)] Дефинираме:
        \begin{equation}
            \begin{split}
                \Gamma(f)(x) = F(f, x)\\
                \tilde g(a) = \mu m \left[ \forall c:\; len(c) = m \& \forall i \leq m:\; (c)_i \leq g(i) \rightarrow u_F(c, a) \neq 0 \right]
            \end{split}
        \end{equation}
        Минимизацията е винаги успешна от предишната точка. Нека $c \leq \langle g(0), g(1), \dots, g(m) \rangle$. За произволно $m$ имаме краен брой кодове на списъци.
    \end{itemize}
\end{proof}

\begin{lemma}[ретракция]\label{lem:retraction}
    Съществуват $\mu$-рекурсивни оператори $T_1, T_2, T_3$ със следните свойства:
    \begin{enumerate}
        \item Ако $f$ представя $\xi \in I$ в смисъла на Гжегорчик
        \begin{equation}
            \abs{\frac{f(k)}{k+1} -\xi} < \frac{1}{k+1}
        \end{equation}
        То $\langle T_1(f), T_2(f), T_3(f)\rangle$ е име на същото $\xi$
        \item За всяка тотална $f$ съществува $\xi \in I$, т.ч. $\langle T_1(f), T_2(f), T_3(f)\rangle$ е име на $\xi$
    \end{enumerate}
\end{lemma}
\begin{proof}
    Строим редица $I_k$ - редица от затворени интервали с рационални краища(може и $\emptyset$):
    \begin{equation}
        \begin{split}
            I_0 = [0; 1] \\
            I_{k+1} = I_k \cap \left[ \frac{f(k)- 1}{k+1}; \frac{f(k) + 1}{k+1} \right]
        \end{split}
    \end{equation}
    Ясно е, че $I_0 \supseteq I_1 \supset I_2 \supset \dots \supset I_k \supset \dots$ и също така $\abs{I_k} \leq \frac{2}{k+1}$

    Намаляваща редица от компактни множества има непразно сечение (топологично съображение).

    Ще си помогнем с:
    \begin{equation}
        \begin{split}
            F(f, 0) = 1\\
            F(f, k+1) = \begin{cases}
                \max I_{k+1} &, I_{k+1} \neq \emptyset \\
                F(f, k) &, I_{k+1} = \emptyset
            \end{cases}
        \end{split}
    \end{equation}
    
    Ще покажем, че за произволна тотална $f$ ще съществува $\xi \in I$:
    \begin{equation}
        \abs{F(f, k+1) - \xi} \leq \frac{2}{k+1}
    \end{equation}
    \begin{itemize}
        \item[(1. случай)] $\bigcap_k I_k \neq \emptyset$. Нека $\xi \in \bigcap_k I_k$. Значи $F(f, k+1)$ е десният край на $I_{k+1}$ и $\xi \in I_{k+1}$, значи $\abs{F(f, k+1) - \xi} \leq \frac{2}{k+1}$
        \item[(2. случай)] $\bigcap_k I_k = \emptyset$. Тъй като интервалите са затворени, има $k \in \N$, т.ч. $I_{k+1} = \emptyset$. Нека най-малкото такова $k$ е $k_0$. Значи:
        \begin{equation}
            F(f, k_0) = F(f, k_0 + 1) = F(f, k_0 + 2) = \dots
        \end{equation}
        Ще дефинираме $\xi = F(f, k_0)$. $\xi \in I_{k_0}$, значи $\xi \in I_0 = [0; 1]$. 

        Да видим грешката:
        \begin{equation}
            \begin{split}
                k \geq k_0 \Rightarrow \abs{F(f, k+1) - \xi} = 0 \leq \frac{2}{k+1} \\
                k \leq k_0 \Rightarrow F(f, k+1) \in I_k,\ \xi \in I_k \xRightarrow{\abs{I_k} \leq \frac{2}{k+1}} \abs{F(f, k+1) - \xi} \leq \frac{2}{k+1}
            \end{split}
        \end{equation}
    \end{itemize}

    Ако $f$ представя $\xi$ в смисъла на Гжегорчик, то сме в случай 1 и резултата е същото $\xi$

    Заключаваме, че операторите $T_1, T_2, T_3$:
    \begin{equation}
        \frac{T_1(f)(k) - T_2(f)(k)}{T_3(f)(k) + 1} = F(f, 2k+3)
    \end{equation}
\end{proof}