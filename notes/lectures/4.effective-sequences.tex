\section{Ефективна сходимост на редици}
\begin{definition}
    $\left\{a_n\right\}_{n=0}^\infty$ клони ефективно към $a$, ако съществува рекурсивна ф-ия $\delta : \N \to \N$ - регулатор на сходимост.
    \begin{equation*}
        \forall k, \forall n:\; n > \delta(k) \Rightarrow |a_n - a| < \frac{1}{k+1}
    \end{equation*}
\end{definition}
\begin{remark}
    Не се интересуваме от изчислимостта на членовете на редицата или крайния елемент
\end{remark}
\begin{example}
    $a_n = a$ - константна редица. $\delta(k)$ - произволна ($> 0$)
\end{example}
\begin{example}
    $a_n = \frac{1}{n+1}$, $\delta(k) = k$
\end{example}
\begin{example}
    $a_n = \sqrt{n+1} - \sqrt{n}$, $\delta(k) = k^2 + 2k$, защото
    \begin{equation*}
        \sqrt{n+1} - \sqrt{n} = \frac{1}{\sqrt{n+1} + \sqrt{n}} < \frac{1}{k+1}
    \end{equation*}
\end{example}
\begin{example}
    $a_n = a_0.q^n, |q| < 1$, проверяваме дали е ефективно сходяща. Нека $h \in Q^{\geq 0}$, т.ч. $|q| < \frac{1}{1+h}$
    \begin{equation*}
        |a_n| = |a|.|q|^n \leq \frac{|a|}{(1+h)^n} \leq \frac{|a|}{1+nh} < \frac{1}{k+1}
    \end{equation*}
    Преобразуваме последното неравенство.
    \begin{equation*}
        \begin{split}
            \frac{|a|}{1+nh} < \frac{1}{k+1}\\
            1+nh > |a|(k+1)\\
            n > \frac{|a|(k+1) - 1}{h}
        \end{split}
    \end{equation*}
    И можем да дефинираме
    \begin{equation*}
        \delta(n) = M(k+1), M > \frac{|a|}{h} \in \N
    \end{equation*}
\end{example}
\begin{example}
    Умножение по константа не променя ефективната сходимост
\end{example}
\begin{example}
    Събиране не променя ефективната сходимост
\end{example}
\begin{lemma}[Ефективните полицаи]
    Нека $\{a_n\}_{n=0}^\infty$, $\{c_n\}_{n=0}^\infty$ - ефективно изчислими и $\{b_n\}_{n=0}^\infty$ е т.ч.
    \begin{equation*}
        \forall n:\; a_n \leq b_n \leq c_n
    \end{equation*}
    Тогава $\{b_n\}_{n=0}^\infty$ е ефективно изчислима.
\end{lemma}
\begin{notation}
    $\overset{\infty}{\forall}n = \exists n'\ \forall n \geq n'$. Интуитивно "вярно от дадено място нататък". В известен (всеки) смисъл е по-силно от обичайния квантор за всеобщност.
\end{notation}
\begin{example}
    $f: \N \to \N$ - тотално изчислима инекция. $range(f)$ е неразрешимо. $a_n = \frac{1}{f(n) + 1}$.

    Фиксираме $k \in \N$. Тогава $A = \left\{n \mid f(n) < k\right\}$ е крайно.

    \begin{equation*}
        \forall k \overset{\infty}{\forall}n f(n) > k
    \end{equation*}
    Нека $n' = max A + 1$. $f(n) \underset{n \to \infty}{\longrightarrow} \infty$ и $a_n \underset{n \to \infty}{\longrightarrow} 0$

    Значи редицата е сходима. Дали е ефективно сходима?

    Нека $\delta(k)$ е регулаторна функция. 
    \begin{equation*}
        n > \delta(k) \Rightarrow \frac{1}{f(n) + 1} < \frac{1}{k+1}
    \end{equation*}
    Да допуснем, че $g(n)$ мажорира $\delta(k)$ за някоя рекурсивна $g$. Тогава:
    \begin{equation*}
        m \in range(f) \iff \exists n \leq g(k):\; m = f(n)
    \end{equation*}
    Но това значи, че $range(f)$ е разрешимо - абсурд. Значи не съществува регулаторна функция, значи не е ефективно сходима границата.
\end{example}