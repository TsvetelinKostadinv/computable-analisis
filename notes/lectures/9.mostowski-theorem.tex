\section{Теорема на Мостовски}
\begin{theorem}[Мостовски \cite{mostowski1957computable}]
    Съществува изчислима редица $\{x_n \in \R\}_{n \in \N}$, т.ч.
    \begin{itemize}
        \item има изчислима $\rho_2: \N^2 \to \N$, т.ч. $\lambda k.\; \rho_2(n,k)$ е представяне на $x_n$ в основа 2
        \item не съществува изчислима $\rho_{10}: \N^2 \to \N$, т.ч. $\lambda k.\; \rho_{10}(n,k)$е представяне на $x_n$ в основа 10
    \end{itemize}
\end{theorem}
\begin{proof}
    Нека $A \& B$ са полуразрешими, рекурсивно неотделими множества ($A \subseteq C$ и $B \subseteq \Bar{C}$, то $C$ не е разрешимо). И $A \cap B = \emptyset$. 
    
    Дефинираме:
    \begin{equation}
        \rho_2(n,k) = \begin{cases}
            0 &, \exists i > k:\; a(i) = n \\
            1 &, \exists i > k:\; b(i) = n \\
            D_k &, \text{иначе}
        \end{cases}
    \end{equation}
    Където $range(a) = A, range(b) = B, \frac{1}{5} = \langle0.D_1 D_2 \dots D_k \dots \rangle_2$.

    Знаем, че $x_n = \sum\limits_{k\in\N} \rho_2(n,k).2^{-k}$ е изчислима редица.

    Нека разгледаме случаи:
    \begin{itemize}
        \item[$n \in A$] $\Rightarrow n = a(i)$ за някое (най-малкото) $i$. Значи $x_n = 0.D_1 D_2 \dots D_i 00 \dots 0 \dots \Rightarrow x_n < \frac{1}{5}$
        \item[$n \in B$] $\Rightarrow n = b(i)$ за някое (най-малкото) $i$. Значи $x_n = 0.D_1 D_2 \dots D_i 11 \dots 1 \dots \Rightarrow x_n > \frac{1}{5}$
        \item[$n \notin A\cup B$] $\Rightarrow x_n = \frac{1}{5}$.
    \end{itemize}

    Допускаме, че $\rho_{10}(n, k)$ съществува - представяне на $x_n$ в основа 10.
    \begin{itemize}
        \item[$n \in A$] $\Rightarrow x_n < \frac{1}{5} \Rightarrow \rho_{10}(n,1) < 2$
        \item[$n \in B$] $\Rightarrow x_n > \frac{1}{5} \Rightarrow \rho_{10}(n,1) \geq 2$
    \end{itemize}
    Значи:
    \begin{equation}
        C = \left\{n \mid \rho_{10}(n,1) < 2 \right\}    
    \end{equation}
    е разрешимо и отделя $A \& B$ - абсурд. Значи няма такава функция $\rho_{10}$.
\end{proof}