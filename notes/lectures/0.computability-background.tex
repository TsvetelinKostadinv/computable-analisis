\section{Формален апарат}
\begin{notation}
    Частична ф-ия $f: \subseteq \N^k \to \N$
\end{notation}
Искаме да дефинираме изчислимост за ф-ия $f: \subseteq \N^k \to \N$
% \begin{definition}
%     Ф-ия $f: \subseteq \N^k \mapsto \N$ е изчислима $\iff$ съществува машина на Тюринг
% \end{definition}

\subsection{Примитивно рекурсивни ф-ии}
\begin{definition}
    Индуктивна дефиниция
    \begin{itemize}
        \item[базови ф-ии] $0(x) = 0, S(x) = x+1, I_k^n(x_1,\dots, x_n) = x_k$
        \item[операции] 
            \subitem Суперпозиция(композиция) - $(f, g_1, \dots, g_k) \mapsto h : h(\mathbf{x}) = f(g_1(\textbf{x}), \dots, g_k(\textbf{x})$
            \subitem Примитивна рекурсия (Ограничен цикъл)
            \begin{equation}
            (f, g) \mapsto h
                \begin{cases}
                    h(\textbf{x}, 0) = f(\textbf{x})\\
                    h(\textbf{x}, y+1) = g(\textbf{x}, y, h(\textbf{x}, y)
                \end{cases}
            \end{equation}
            \subitem Минимизация - $f \mapsto h : h(\textbf{x}) = \mu y(f(\textbf{x}, y) = 0)$
    \end{itemize}
    Където $\textbf{x}$ е вектор $(x_1, \dots, x_n)$

    $f$ е примитивно рекурсивна $\iff f\in Close($базови ф-ии$ \mid $ суперпозиция и примитивна рекурсия$)$ 
\end{definition}
\begin{definition}
    $f$ е частично рекурсивна(изчислима) $\iff f\in Close($базови ф-ии$ \mid $ суперпозиция, примитивна рекурсия и минимизация$)$ 
\end{definition}
\begin{definition}
    тотална \& изчислима $\iff$ рекурсивна
\end{definition}

\begin{notation}
    $h(\textbf{x})\downarrow \iff \forall z < y f(\textbf{x}, y)\downarrow \neq 0 \land f(\textbf{x}, y) = 0$
\end{notation}
Не е очевидно дали всяка изчислима функция има такова представяне - чрез суперпозиция, примитивна рекурсия и минимизация.
\begin{proposition}
    Има функция, която е рекурсвина, но не е примитивно рекурсивна.
\end{proposition}
\begin{proof}
    Доказва се по "Изчислимост и сложност"
\end{proof}
Няколко примера за примитивно рекурсивно ф-ии:
\begin{example}
    \begin{equation*}
        \lambda x y.\ x+y
    \end{equation*}
\end{example}
\begin{example}
    \begin{equation*}
        \lambda x y.\ x*y
    \end{equation*}
\end{example}
\begin{example}
    \begin{equation*}
        \lambda x y.\ x^y
    \end{equation*}
\end{example}
\begin{example}
    \begin{equation*}
        \lambda x.\ x!
    \end{equation*}
\end{example}
\begin{example}
    Отсечена разлика
    \begin{equation*}
        \lambda x y.\ x\dot{-}y = max\{x-y, 0\}
    \end{equation*}
\end{example}
\begin{example}
    \begin{equation*}
        \lambda x y.\ min\{x, y\}
    \end{equation*}
\end{example}
\begin{example}
    \begin{equation*}
        \lambda x y.\ max\{x, y\}
    \end{equation*}
\end{example}
\begin{example}
    \begin{equation*}
        \lambda x y.\ |x - y|
    \end{equation*}
\end{example}
\begin{example}
    \begin{equation*}
        \lambda x.\ sg(x) = \begin{cases}
            0, x = \text{0} \\
            1, x \neq \text{0}
        \end{cases}
    \end{equation*}
\end{example}
\begin{example}
    \begin{equation*}
           \lambda x.\ \Bar{sg(x)} = 1 - sg(x)
    \end{equation*}
\end{example}

Конструкция за случаите
\begin{equation}
    f(\textbf{x}) = \begin{cases}
        g_1(\textbf{x}), P_1(\textbf{x}) \\
        \vdots \\
        g_k(\textbf{x}), P_k(\textbf{x})
    \end{cases}
\end{equation}
Изразява се чрез примитивна рекурсия

\begin{definition}
    $P$ е примитивно рекурсивен(изчислим) предикат $\iff$ $\xi_p: \N^k \mapsto \{0, 1\}$ е примитивно рекурсивна(изчислима) ф-ия, където $\xi_P(\textbf{x}) = 0$, ако $P(\textbf{x})$ и $xi_P(\textbf{x}) = 1$, ако $\neg P(\textbf{x})$ 
\end{definition}

\subsection{Ограничени операции}
Ограничена сума:
\begin{equation}
    f \mapsto \lambda \textbf{x}y \sum\limits_{z \leq y} f(\textbf{x}, z)
\end{equation}
Ограничено произведение, минимум, максимум

Ограничена минимизация
\begin{equation}
    f \mapsto \lambda \textbf{x}y \mu z \leq y (f(\textbf{x}, z) = 0
\end{equation}

Ограничените операции са \textbf{изчислими}

Ограничен квантор за изчислим предикат:
\begin{equation}
    P \mapsto \square z \leq y P(\textbf{x}, z) \text{, където } \square \in \{\forall, \exists\}
\end{equation}

\section{Кодиране}
Кодиране на $\N^k$. Започваме от $n=2$
\begin{equation}
    J: \N^2 \to \N
    \begin{cases}
        J(x, y) = 2^x(2y+1) \dot{-} 1 \\
        J(x, y) = 1 + \dots + x + y + x
    \end{cases}
\end{equation}
И двата варианта вършат работа. И двете са \textbf{примитивно рекурсивни биекции}.

Забелязваме, че $J(x, y) \geq max\{x, y\}$.

Нека $L, R$ са 2 декодиращи ф-ии, т.ч:
\begin{equation}
    \begin{split}
    L(J(x,y)) = x \\
    R(J(x,y)) = y \\
    J(L(z), R(z)) = z
\end{split}
\end{equation}
Защо $L, R$ са примитивно рекурсивни?
\begin{equation}
    \begin{split}
        L(z) = \mu_{x \leq z}\; \exists y \leq z (z = J(x,y)) \\
        R(z) = \mu_{y \leq z}\; \exists x \leq z (z = J(x,y)) \\
    \end{split}
\end{equation}

И ще кодираме с индуктивна дефиниция:
\begin{itemize}
    \item[база] $J_1(x) = x$
    \item[стъпка] $J_{k+1}(x_1, \dots, x_k, x_{k+1}) = J(J_k(x_1, \dots, x_k), x_{k+1})$
\end{itemize}
Лесно се вижда, че $J_k : \N^k \to \N$ е примитивно рекурсивна биекция

За обратните функции
\begin{equation*}
    (J_k(\textbf{x}))_i^k = x_i
\end{equation*}

\begin{lemma}[Гьодел]
    Съществува примитивно рекурсивна ф-ия $\gamma: \N^2 \to \N$, т.ч. всяка крайна непразна редица от ест. числа има вида $\gamma(z, 0), \gamma(z, 1), \dots, \gamma(z, t)$ за подходящо $z$
\end{lemma}
\begin{proof}
    Ще гледаме на $\gamma$ като редица $s_0, s_1, \dots, s_t$ (деф. $s_i = \gamma(z, i)$ 
    \begin{equation*}
        z = J_{t+2}(0, s_t, s_{t-1}, \dots, s_0)
    \end{equation*}
    И дефиницията на $s_i$
    \begin{equation*}
        \begin{split}
            s_0 = R(z) \\
            s_1 = R(L(z)) \\
            \vdots \\
            s_t = R(L^t(z))
        \end{split}
    \end{equation*}
    значи $\gamma(z, i) = R(L^i(z))$ и това е дефиницията. Но $L \& R$ са примитивно рекурсивни $\Rightarrow$ $\gamma$ е примитивно рекурсивна
\end{proof}

\begin{definition}
    $\psi : \N^{k} \to \N$ е универсална ф-ия за примитивно рекурсивните ф-ии $\iff \forall f \in PrimRec \exists e \in \N \forall \textbf{x} \in \N^k: f(\textbf{x}) = \psi(e, \textbf{x})$ 
\end{definition}

\begin{proposition}
    Съществува рекурсивна ф-ия $\psi: \N^2 \to \N$, която е универсална за класа на примитивно рекурсивните едноместни функции. 
\end{proposition}
Зависи от формализма - ако бяхме с машини на Тюринг, има естествено кодиране и $\psi$ само симулира действието на машината. С примитивно рекурсивни ф-ии ще трябва да кодираме синтактичните им дървета.

\begin{proof}
    \begin{equation*}
        \begin{split}
            0 & \mapsto \langle 0, c_0\rangle \\
            s & \mapsto \langle 0, c_1\rangle \\
            I_k^n &\mapsto \langle 0, \langle 1, n, k\rangle\rangle \\
            \underbrace{f}_{e_f}, \underbrace{g_1}_{e_1}, \dots, \underbrace{g_k}_{e_k} \overset{\text{примитивна рекурсия}}{\mapsto} h &\mapsto \langle 1, \langle1, \langle e_f, e_1, \dots, e_k\rangle\rangle\rangle \\
            \underbrace{f}_{e_f}, \underbrace{g}_{e_g} \overset{\text{композиция}}{\mapsto} h &\mapsto \langle 2, \langle e_f, e_g\rangle\rangle
        \end{split}
    \end{equation*}
    Представлява кодировка на примитивно рекурсивните ф-ии.
\end{proof}
\begin{proposition}
    $\psi$ не е примитивно рекурсивна
\end{proposition}
\begin{proof}
    Интуитивно е подобно на halting-problem.

    Допускаме, че е примитивно рекурсивна, тогава има код. Тогава нека следната ф-ия
    \begin{equation*}
        \lambda x.\ \psi(x, x) + 1
    \end{equation*}
    ще има код $e$. Тогава
    \begin{equation*}
        \psi(e, x) = \psi(x, x) + 1
    \end{equation*}
    Но при $x = e$ получаваме абсурд.
\end{proof}

\begin{theorem}
    За всяка частично рекурсивна ф-ия $f: \subseteq \N^k \to \N$, съществува примитивно рекурсивна ф-ия $h: \N^{k+2} \to \N$, т.ч. 
    \begin{equation}
        f(\textbf{x}) = y \iff \exists z (h(\textbf{x}, y, z) = 0)
    \end{equation}
\end{theorem}
\begin{proof}
    Индукция по дефиницията на $f$.
    \begin{itemize}
        \item[база] Ако $f$ е примитивно рекурсивна, то $h(\textbf{x}, y, z) = |f(\textbf{x}) - y|$
        \item[операции]
            \subitem[суперпозиция] $f(\textbf{x}) = f_0(f_1(\textbf{x}), \dots, f_k(\textbf{x}))$. При инд. предположение:
            \begin{equation*}
                \begin{split}
                    f_0(\textbf{t}) = y \iff \exists z_0 (h_0(\textbf{t}, y, z_0) = 0\\
                    f_i(\textbf{x}) = t_i \iff \exists z_i (h_i(\textbf{t}, y, z_i) = 0 \text{ за } i \in \{1, \dots k\}
                \end{split}
            \end{equation*}
            Тогава
            \begin{equation*}
            \begin{split}
                & f(\textbf{x}) = y \\
                & \iff \exists t_1, \dots t_k (\&_{i=1}^{k} f_i(\textbf{x}) = t_i \& f_0(\textbf{t}) = y) \\
                & \iff \text{пропуснах реда, естествена трансформация е}\\
                & \iff \exists t_1 \dots t_k, z_0 \dots z_k (h_1(\textbf{x}, t_1, z_1) + \dots + h_k(\textbf{x}, t_k, z_k) + h_0(\textbf{t}, y, z_0) = 0)\\
                & z = J_{2k+1}(t_1, \dots, t_k, z_0, z_1, \dots z_k) \\
                & \iff \exists z (\sum\limits_{i=1}^{k} h_i(\textbf{x}, (z)_i, (z)_k+i+1) + h_0(\textbf{t}, (z)_0, (z)) 
            \end{split}
            \end{equation*}
            \subitem[примитивна рекурсия] 
            \begin{equation*}
                \begin{split}
                    f(\textbf{x}, 0) = f_0(\textbf{x}) \\
                    f(\textbf{x}, y+1) = f_1(\textbf{x}, y, f(\textbf{x}, y))
                \end{split}
            \end{equation*}
            С индукционно предположение:
            \begin{equation*}
                \begin{split}
                    f_0(\textbf{x}) = y \iff \exists z_0 (h_0(\textbf{x}, y, z_0) = 0) \\
                    f_1(\textbf{x}, y, t) = t' \iff \exists z_1 (h_1(\textbf{x}, y, t, t', z_1) = 0) \\
                \end{split}
            \end{equation*}
            Тогава
            \begin{equation*}
                \begin{split}
                    & f(\textbf{x}, y) = z \\ 
                    & \iff \exists z_0, \dots z_y (f_0(\textbf{x}) = z_0 \& \forall i < y z_{i+1} = f_1(\textbf{x}, i, z_i) \& z_y = z) \\
                    & \iff \exists z_0, \dots z_y (\exists w_0 h_0(\textbf{x}, z_0, w_0) = 0 \& \forall i < y \exists w_{i+1} (f_1(\textbf{x}, i, z_i, z_{i+1}, w_{i+1}) = 0) \& z_y = z) \\
                    & z \text{ е код на } z_0, \dots z_y, w_0, w_1, \dots, w_y \\
                    & \iff \exists z ( h_0(\textbf{x}, \gamma(z, 0), \gamma(z, y+1) + \sum\limits_{i<y} h_1(\textbf{x}, i, \gamma(z, i), \gamma(z, i+1), \gamma(z, i+y+1) + |\gamma(z, y) - z| = 0 )
                \end{split}
            \end{equation*}
            \subitem[минимизация]
            \begin{equation*}
                f(\textbf{x}) = \mu z (f_0(x, z) = 0)
            \end{equation*}
            С инд. предположение:
            \begin{equation*}
                f_0(\textbf{x}, z) = t \iff \exists w (h_0(\textbf{x}, z, t) = 0)
            \end{equation*}
            \begin{equation*}
                \begin{split}
                    & f(x) = y \\
                    & \iff \exists t_0 \dots t_y ( \forall i \leq y f_0(x, i) = t_i \& \forall i < y t_i \neq 0 \& t_y = 0) \\
                    & \iff \exists t_0 \dots t_y ( \forall i\leq y \exists w_i (h_0(\textbf{x}, i ,t_i, w_i) = 0) \& \forall i < y t_i \neq 0 \& t_y = 0 ) \\
                    & z \text{ е код на } t_0 \dots t_y, w_0 \dots w_y\\
                    & \iff \exists z (\sum\limits_{i \leq y} h_0(\textbf{x}, i, \gamma(z, i), \gamma(z, i+y+1) + \sum\limits_{i < y} \Bar{sg(\gamma(z,i)} + \gamma(z, y) = 0 )
                \end{split}
            \end{equation*}
    \end{itemize}
\end{proof}
\begin{theorem}[Нормална форма] \label{thm:normal-form}
    За вс. частично рек. ф-ия $f$ съществува примитивно рек. $h: \N^{k+1} \to \N$, т.ч.
    \begin{equation}
        f(\textbf{x}) = L( \mu z (h(\textbf{x}, z) = 0 )
    \end{equation}
\end{theorem}
\begin{proof}
    Нека $h'$ е т.ч.
    \begin{equation*}
        f(\textbf{x}) = y \iff \exists t(h'(\textbf{x}, y, t) = 0)
    \end{equation*}
    Нека $z = J(y, t)$ и значи търсената $h$ е
    \begin{equation*}
        h(\textbf{x}, z) = h'(\textbf{x}, L(z), R(z))
    \end{equation*}
    И от тук следва, че:
    \begin{equation*}
        f(\textbf{x}) \downarrow \iff \exists z(h(\textbf{x}, z) = 0)
    \end{equation*}
\end{proof}

Как от универсалната ф-ия $\psi$ получаваме ефективна номерация на всяка $k$-местна изчислима ф-ия?
\begin{equation}
    \varphi_e^{(k)}(\textbf{x}) = L(\mu t (\psi(e, J_{k+1}(\textbf{x}, t)) = 0) )
\end{equation}
\begin{equation*}
    g(y) = h( (y)_1, \dots, (y)_{k+1})
\end{equation*}
и $e$ е код на $g$
\begin{equation*}
    \psi(e, y) = g(y)
\end{equation*}

\newpage
\section{Разрешими/Полуразрешими множества}
От 2024-03-06.
\subsection{Преговор}
\begin{theorem}
    Теорема за нормалната форма.
    \begin{equation*}
        \varphi_e^{(k)}(\textbf{x}) = L( \mu t( \psi(e, J_{k_1}( \textbf{x}, t )) = 0 ))
    \end{equation*}
    \thref{thm:normal-form}
\end{theorem}
\begin{notation}
    \begin{equation*}
        \Phi(e, x) = \varphi_e^1(x)
    \end{equation*}
    е частично рекурсивна.
\end{notation}
При $\Phi(e,x)$ не върви диагонализацията, защото просто не е дефинирана при собствения си код.
\begin{proposition}
    Ф-ията $\lambda x. \Phi(x, x)$ няма тотално изчислимо продължение
\end{proposition}
\begin{proof}
    Да допуснем, че име - $\Theta$ е изчислимо продължение на $\Phi(x, x)$. Нека $e$ е индекс на $\lambda x. \Theta(x) + 1$.

    Значи $\Phi(e, x) = \Theta(x) + 1 \forall x$

    при $x=e:\; !\Phi(e,e) \Rightarrow \Theta(e) = \Phi(e, e)$ - $\Theta$ продължава $\Phi$

    Тогава $\Phi(e, e) = \Theta(e) + 1$
\end{proof}
\begin{corollary}
    \begin{equation*}
        K = \left\{x \mid \Phi(x, x) \downarrow \right\}
    \end{equation*}
\end{corollary}
\begin{proof}
    Ако допуснем, че $K$ е разрешимо.
    \begin{equation*}
        \Theta(x) = \begin{cases}
            \Phi(x, x) & , x \in K \\
            0 & , x \notin K
        \end{cases}
    \end{equation*}
    Значи $\Theta$ е рекурсивно(тотално и изчислимо) продължение на $\Phi$
\end{proof}

\subsection{Разрешими/Полуразрешими множества}
\begin{definition}
    \begin{equation*}
        R \subseteq \N^k
    \end{equation*}
    е полуразрешимо(рекурсивно номеруемо) $\iff$ ако $R = dom(f)$, където $f: \subseteq \N^k \to \N$ - изчислима
\end{definition}
\begin{corollary}
От доказаните до момента, можем да видим свойства на полуразрешимите мн-ва:
    \begin{enumerate}
        \item $R$ е полуразрешимо $\iff$ съществува примитивно рек. ф-ия $h: \N^{k+1} \to \N$, т.ч. $\left(\textbf{x} \in R \iff \exists t( h(\textbf{x}, t ) = 0)\right)$
        
        \begin{proof}
        \begin{itemize}
            \item[($\Rightarrow$)] $R=dom(f)$ по \thref{thm:normal-form} $f(\textbf{x}) = L(\mu t (h(\textbf{x}, t) = 0))$

            $\textbf{x} \in R \iff f(\textbf{x})\downarrow \iff \exists t (h(\textbf{x}, t) = 0)$

            \item[($\Leftarrow$)] имаме примит. рек. ф-ия $h$, трябва да докажем, че $R$ е домейн на някаква ф-ия, която е изчислима. Нека
            \begin{equation*}
                f(\textbf{x}) = \mu t ( h( \textbf{x}, t ) = 0 )
            \end{equation*}
        \end{itemize}
        \end{proof}
        \item $R$ е полуразрешимо $\iff$ $\exists f: \subseteq \N^k \to \N$, т.ч. $R = range(f)$
        
        \begin{proof}
        \begin{itemize}
            \item[($\Rightarrow$)] $R = dom(f)$ е полуразрешимо и $f: \subseteq \N \to \N$. Нека $h$ е примитивно рекурсивна, т.ч.
            \begin{equation*}
                f(x) = y \iff \exists t ( h(x, y, t) = 0 ) 
            \end{equation*}
            И дефинираме
            \begin{equation*}
                g(x) = \begin{cases}
                    x &, f(x) \downarrow \\
                    \uparrow &, f(x) \uparrow
                \end{cases}
            \end{equation*}
            Интуитивно изравняваме домейна и кодомейна в $g$ като подфункция на идентитета.

            Тогава:
            \begin{equation*}
                x \in range(g) \iff x \in dom(f)
            \end{equation*}
            \item[($\Leftarrow$)] $x \in range(g), g: \subseteq \N \to \N$ е изчислима

            избираме $h: \N^3 \to N$ е примитивно рек, т.ч.
            \begin{equation*}
                g(x) = y \iff \exists z ( h(x, y, z) = 0 )
            \end{equation*}
            \begin{equation*}
                y \in R \iff \exists x g(x) = y \iff \exists x \exists z h(x, y, z) = 0 \iff \exists t ( h((t)_0, y, (t)_1) = 0 )
            \end{equation*}
            $R$ е полуразрашимо от горното
        \end{itemize}
        \end{proof}
        \item $R$ е разрешимо $\iff R$ е полуразрешимо
        
        \begin{proof}
            $R = dom(f)$
            \begin{equation*}
                f(\textbf{x}) = \begin{cases}
                    0 &, \textbf{x} \in R \\
                    \uparrow &, \textbf{x} \notin R
                \end{cases}
            \end{equation*}
        \end{proof}
        \item $K$ е полуразрешимо и \textbf{не е} разрешимо
        \item $R_1, R_2$ - полуразрешими $\Rightarrow R_1 \cap R_2 \& R_1 \cup R_2$ са полуразрешими
        
        \begin{proof}
            \begin{itemize}
                \item[$\cap$] тривиално е
                \item[$\cup$] избираме $h_1, h_2$, т.ч
                \begin{equation*}
                    \textbf{x} \in R_i \iff \exists t ( h_i(\textbf{x}, t) = 0 ),\; i \in \{1, 2\}
                \end{equation*}
                \begin{equation*}
                    \textbf{x} \in R_1 \cup R_2 \iff \exists ( h_1( \textbf{x}, t ).h_2(\textbf{x}, t) = 0 )
                \end{equation*}
            \end{itemize}
        \end{proof}
    \end{enumerate}
\end{corollary}
\begin{theorem}[Пост]\label{thm:Post}
    $R$ е разрешимо $\iff$ $R \& \Bar{R}$ са полуразрешими
\end{theorem}
\begin{proof}
    \begin{itemize}
        \item[$\Rightarrow$] $R$ е резрешимо $\Rightarrow$ $R$ е полуразрешимо и $\Bar{R}$ е разрешимо(в частност полуразрешимо)
        \item[$\Leftarrow$] Знаем, че
        \begin{equation*}
            \begin{split}
                & \textbf{x} \in R \iff \exists t( h(\textbf{x},t) = 0 )\\
                & \textbf{x} \in \Bar{R} \iff \exists t( h'(\textbf{x},t) = 0 )
            \end{split}
        \end{equation*}
        Тогава нека:
        \begin{equation*}
            g(\textbf{x}) = \mu t ( h(\textbf{x},t).h'(\textbf{x}, t) = 0 )
        \end{equation*}
        Очевидно $g$ е рекурсивна и твърдим, че:
        \begin{equation*}
            \textbf{x} \in R \iff h(\textbf{x}, g(\textbf{x})) = 0
        \end{equation*}
        Нека $\textbf{x} \in R$. Допускаме, че $h(\textbf{x}, g(\textbf{x})) \neq 0 \Rightarrow h'(\textbf{x}, g(\textbf{x})) = 0 \Rightarrow x \in \Bar{R}$ Абсурд
    \end{itemize}
\end{proof}
