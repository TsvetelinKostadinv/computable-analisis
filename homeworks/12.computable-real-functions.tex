% \begin{proposition}
%     $n,m \mapsto \sqrt{\frac{n+1}{m+1}}$ е изчислима функция.
% \end{proposition}
% \begin{proof}
%     Постъпково:
%     \begin{itemize}
%         \item[(1. стъпка)] Разглеждаме $n \mapsto \sqrt{1 + \frac{1}{n}} : \N^+ \to \R$.
%               \begin{equation}
%                   \sqrt{1+x} = 1 + \frac{1}{2}x - \frac{1}{8}x^2 + \frac{1}{16}x^3 - \frac{5}{128}x^4 + \dots = \sum_{n \in \N} (-1)^n \frac{(2n)!}{(1-2n).(n!)^2.4^n}x^n
%               \end{equation}
%               за $x \in [-1; 1]$

%               Тогава:
%               \begin{equation}
%                   \begin{split}
%                       \sqrt{1 + \frac{1}{n}} = \sum_{s \in \N^+} (-1)^s \frac{(2s)!}{(1-2s).(s!)^2.4^s.n^s} \\
%                       = \sum_{s=1}^{t+1} \frac{(-1)^s (2s)!}{(1-2s).(s!)^2.4^s.n^s} + \sum_{s = t+2}^\infty \frac{(-1)^s (2s)!}{(1-2s).(s!)^2.4^s.n^s}
%                   \end{split}
%               \end{equation}
%               За грешката имаме?:
%               \begin{equation}
%                   \left|\sum_{s = t+2}^\infty \underbrace{ \frac{(-1)^s(2s)!}{(1-2s).(s!)^2.4^s.n^s}}_{\text{монотонно намаляваща}} \right| \overset{\text{Лайбниц}}{\leq} \frac{(2t+4)!}{(2t +3)((t+2)!)^2.4^{t+2}.n^{t+2}} < \frac{1}{t+1}
%               \end{equation}

%         \item[(2. стъпка)] Разглеждаме $n \mapsto \sqrt{n+1} : \N \to \R$.
%               \begin{equation}
%                   \sqrt{n+1} = \sum_{m=1}^n \sqrt{m+1} - \sqrt{m} = \underbrace{\sum_{m=0}^n \sqrt{m} . \left( 1 + \frac{1}{\sqrt{m+1}}\right)}_{\text{изчислима сума}}
%               \end{equation}
%               Значи $n \mapsto \log\left(n+1\right)$ е изчислима функция.

%         \item[(3. стъпка)] Разглеждаме $m, n \mapsto \sqrt{\frac{m+1}{n+1}} = \frac{\sqrt{m+1}}{\sqrt{n+1}} : \Q \to \R$ - също е изчислима. Има $a, b, c$:
%               \begin{equation}
%                   \left|\frac{a(m,n,t) - b(m,n,t)}{c(m,n,t) + 1} - \sqrt{\frac{m+1}{n+1}}\right| < \frac{1}{t+1}
%               \end{equation}
%     \end{itemize}
% \end{proof}

\begin{problem}
Докажете, че реалната функция $\xi \to \sqrt{\xi}$ е изчислима в $(0, +\infty)$ с метода на трисекцията.
\end{problem}
\begin{solution}
    Интуитивно - $\sqrt{x}$ е монотонно растяща непрекъсната функция, значи е изчислима.

    Ще докажем за $\xi \in (0; 1]$, за $\xi' > 1$ можем да направим трансформацията $x \to \frac{1}{x}$, която е биекция м/у $(0;1]$ и $[1; \infty)$.

    Нека за $\xi$ имаме име $\langle f, g, h \rangle$.

    Ще дефинираме индуктивно редици $\{u_k\}_{k \in \N}$ и $\{v_k\}_{k \in \N}$ от рационални числа, т.ч.:
    \begin{equation}
        \begin{split}
            0 < u_k \leq u_{k+1} < v_{k+1} \leq v_k \\
            v_{k+1} - u_{k+1} \leq \frac{2}{3}(v_k - u_k)
        \end{split}
    \end{equation}
    и допълнително:
    \begin{equation}
        u_k^2 < \xi < v_k^2
    \end{equation}
    Което ще ни гарантира:
    \begin{equation}
        u_k < \sqrt{\xi} < v_k
    \end{equation}
    \begin{itemize}
        \item[(База)] Дефинираме $u_0, v_0$. Искаме:
            \begin{equation}
                \begin{split}
                    0 < u_0 < v_0 \\
                    u_0 < \sqrt{\xi} < v_0
                \end{split}
            \end{equation}
            Ще ни свършат работа:
            \begin{equation}
                \begin{split}
                    v_0 = |f(0) - g(0)| + 1 \geq 1 \\
                    u_0 = \frac{1}{v_0} \leq 1
                \end{split}
            \end{equation}
            Очевидно $0 < u_0 < v_0$ и за произволно $\xi \in (0; 1] \subseteq [u_0^2;v_0^2]$ имаме $u_0^2 < \xi < v_0^2$
        \item[(Стъпка)] Нека са дефинирани $u_k\ \&\ v_k$ и изпълняват неравествата. Разделяме $[u_k;v_k]$ на 3 части с еднаква дължина:
            \begin{equation}
                \begin{split}
                    \tilde{u_k} = \frac{2u_k + v_k}{3} \\
                    \tilde{v_k} = \frac{u_k + 2v_k}{3} \\
                    \log \tilde{u_k} < \log \tilde{v_k}
                \end{split}
            \end{equation}
            Нека $a, b, c$ са т.ч.:
            \begin{equation}
                \forall n,\ m,\ t:\; \abs{\left(\frac{m+1}{n+1}\right)^2 - \frac{a(m, n, t) - b(m, n, t)}{c(m, n, t) + 1}} <\frac{1}{t+1}
            \end{equation}
            \begin{equation}
                \begin{split}
                    a = a(m,n,t)\\
                    b = b(m,n,t)\\
                    c = c(m,n,t)\\
                    a' = a(m',n',t)\\
                    b' = b(m',n',t)\\
                    c' = c(m',n',t)\\
                    \frac{m+1}{n+1} = \tilde{u_k} \\
                    \frac{m'+1}{n'+1} = \tilde{v_k} \\
                    \lim_{t\to\infty} \max \left( \left| \frac{a - b}{c + 1} - \frac{f(t)-g(t)}{h(t) + 1} \right|,\left| \frac{a' - b'}{c' + 1} - \frac{f(t)-g(t)}{h(t) + 1} \right| \right) \\
                    = \max \left(|\tilde u_k^2- \xi|, |\tilde v_k^2- \xi|\right) > 0
                \end{split}
            \end{equation}
            Нека:
            \begin{equation}
                \begin{split}
                    t_k = \mu\ t:\; \left[\max\left(\abs{\frac{a - b}{c+1} - \frac{f(t)-g(t)}{h(t)+1}}, \abs{\frac{a' - b'}{c'+1} - \frac{f(t)-g(t)}{h(t)+1}}\right) \geq \frac{2}{t+1}\right] \\
                    a_k = a(m,n,t_k)\\
                    b_k = b(m,n,t_k)\\
                    c_k = c(m,n,t_k)\\
                    a_k' = a(m',n',t_k)\\
                    b_k' = b(m',n',t_k)\\
                    c_k' = c(m',n',t_k)\\
                    f_k = f(t_k) \\
                    g_k = g(t_k) \\
                    h_k = h(t_k)
                \end{split}
            \end{equation}
            \begin{itemize}
                \item[(1 сл.)] Нека $\frac{a_k - b_k}{c_k + 1} - \frac{f_k - g_k}{h_k} \geq \frac{2}{t_k+1}$.
                
                Да разгледаме:
                \begin{equation}
                    \tilde u_k^2 - \xi > \frac{a_k - b_k}{c_k+1} - \frac{1}{t_k+1} - \left(\frac{f_k-g_k}{h_k + 1} + \frac{1}{t_k +1}\right) \geq 0
                \end{equation}
                Значи $\tilde u_k^2 > \xi$
    
                Избираме:
                \begin{equation}
                    \begin{split}
                        u_{k+1} = u_k \\
                        v_{k+1} = \tilde u_k
                    \end{split}
                \end{equation}
    
                \item[(2 сл.)] Нека $\frac{f_k - g_k}{h_k} - \frac{a_k - b_k}{c_k + 1} > \frac{2}{t_k+1}$.
                
                Да разгледаме:
                \begin{equation}
                    \xi - \tilde u_k^2 > \frac{f_k - g_k}{h_k} - \frac{a_k - b_k}{c_k + 1} > 0
                \end{equation}
                Значи $\tilde u_k^2 < \xi$
    
                Избираме:
                \begin{equation}
                    \begin{split}
                        u_{k+1} = \tilde u_k \\
                        v_{k+1} = v_k
                    \end{split}
                \end{equation}
            \item[(3. сл.)] Нека $\frac{a_k' - b_k}{c_k' + 1} - \frac{f_k - g_k}{h_k} \geq \frac{2}{t_k+1}$.
                
                Значи $\tilde v_k^2 > \xi$
    
                Избираме:
                \begin{equation}
                    \begin{split}
                        u_{k+1} = u_k \\
                        v_{k+1} = \tilde v_k
                    \end{split}
                \end{equation}
            \item[(4. сл.)] Нека $\frac{a_k' - b_k}{c_k' + 1} - \frac{f_k - g_k}{h_k} \geq \frac{2}{t_k+1}$.
                
                Значи $\tilde v_k^2 < \xi$
    
                Избираме:
                \begin{equation}
                    \begin{split}
                        u_{k+1} = \tilde v_k \\
                        v_{k+1} = v_k
                    \end{split}
                \end{equation}
            \end{itemize}
        $u_k, v_k$ са $\mu$-рекурсивни функционали над $f,g,h,k$.
    \end{itemize}
\end{solution}

\begin{problem}
Докажете, че $arctg : [-1, 1] \to \R$ е изчислима реална функция, като използвате нейното развитие в ред на Тейлър.
\end{problem}
\begin{solution}
    Ще използваме развитието:
    \begin{equation}\label{eq:arctg-taylor}
        \arctg(x) = \sum_{k\in\N} \frac{(-1)^k x^{2k+1}}{2k+1}
    \end{equation}
    Реда е сходим (в класически смисъл) за $|x| \leq 1$.

    Ще търсим $\mu$-рекурсивни оператори $\langle F, G, H \rangle$, т.ч. по дадено име на $x \in [-1, 1]$ - $\langle f, g, h\rangle$ произвеждат име на $\arctg(x)$ - $\langle F(f, g, h), G(f, g, h), H(f, g, h)\rangle$.

    \begin{itemize}
        \item[($\arctg\frac{n - m}{p+1}$)] Нека $n,m,p \in \N$ и са такива, че $q = \frac{n - m}{p+1} \in [-1; 1]$. Ще докажем, че $\arctg\frac{n - m}{p+1} : \N^3 \to \R$ е изчислима функция

            Разглеждаме абсолютното частното на $s+1$ и $s$-тия член на \eqref{eq:arctg-taylor} при $x = q$:
            \begin{equation}
                \abs{\frac{\frac{(-1)^{s+1} q^{2s+3}}{2s+3}}{\frac{(-1)^s q^{2s+1}}{2s+1}}} = \abs{- q^2 \frac{2s+1}{2s + 3}} = q^2 \frac{2s+1}{2s + 3}
            \end{equation}

            Ще търсим $s_0(q) = \mu\ s:\; q^2 \frac{2s+1}{2s+3} \leq 1$ (минимизацията винаги завършва, защото $|q| \leq 1$ и дробта също $<1$) и ще имаме:
            \begin{equation}
                \forall s \geq s_0(q):\; \frac{(-1)^{s+1} q^{2s+3}}{2s+3} \leq \frac{(-1)^s q^{2s+1}}{2s+1}
            \end{equation}

            \begin{equation}
                \abs{\arctg q - \sum_{s=0}^{s_0(q) + t} \frac{(-1)^s q^{2s+1}}{2s+1}} = \abs{\sum_{s_0(q) + t + 1}^{\infty} \frac{(-1)^s q^{2s+1}}{2s+1}}
            \end{equation}
            Да уточним защо $\left\{\frac{q^{2s+1}}{2s+1}\right\}_{s = s_0(q) + t + 1}^\infty$ е монотонно намаляваща редица - $|q| \leq 1$, а знаменателят е линейна функция.
            \begin{equation}
                \abs{\sum_{s_0(q) + t + 1}^{\infty} \frac{(-1)^s q^{2s+1}}{2s+1}} \overset{\text{Лайбниц}}{\leq} \frac{q^{2s_0(q) + 2t + 2}}{2s_0(q) + 2t + 3} \overset{\delta(t) = ?}{<} \frac{1}{i+1}
            \end{equation}
            Искаме грешката да е по-малка от $\frac{1}{i+1}$ за всяко $i$, където $t$ се определя от $n,m,p,i$.
            \begin{equation}
                t \geq \underbrace{i + 1 \dot - s_0(q) \dot - 3}_{=\delta(n,m,p,i)}
            \end{equation}
            \begin{equation}
                \frac{ q^{2s_0(q) + 2t + 2}}{2s_0(q) + 2t + 3} \leq \frac{q^{2i - 2}}{2i + 5} \leq \frac{1}{2i+5} < \frac{1}{i+1}
            \end{equation}
            Значи за $q \in [-1; 1]$ $\arctg q$ е изчислима функция.
        \item[($\arctg\xi$)] Нека $\xi \in [-1; 1]$ е реално число. И $a, b, c : \N^4 \to \N$ са тотални, изчислими и т.ч.:
            \begin{equation}
                \forall i \in \N:\; \abs{\arctg\frac{n - m}{p+1} - \frac{a(n,m,p,i) - b(n,m,p,i)}{c(n,m,p,i) + 1}} < \frac{1}{i+1}
            \end{equation}
            Нека $\xi \in [-1; 1]$, $\langle f, g, h \rangle$ - негово име и дефинираме:
            \begin{equation}
                \begin{split}
                    F(f, g, h)(t) = a(f(2t+1), g(2t+1), h(2t+1), 2t+1) \\
                    G(f, g, h)(t) = b(f(2t+1), g(2t+1), h(2t+1), 2t+1) \\
                    H(f, g, h)(t) = c(f(2t+1), g(2t+1), h(2t+1), 2t+1)
                \end{split}
            \end{equation}
            В предишният случай полагаме $q = \frac{f(2t+1) - g(2t+1)}{h(2t+1) + 1}$.
            \begin{equation}
                \begin{split}
                    \abs{\arctg\xi - \frac{F(f,g,h)(t) - G(f,g,h)(t)}{H(f,g,h)(t) + 1}}                                   \\
                    \leq \abs{\arctg\xi - \arctg q } + \abs{\arctg q - \frac{F(f,g,h)(t) - G(f,g,h)(t)}{H(f,g,h)(t) + 1}} \\
                    \overset{\substack{\text{Т-ма за кр. нар.}                                                            \\ \exists \nu \in [-1; 1]}}{=} \abs{\frac{1}{\nu^2 + 1}(\xi - q)} + \abs{\arctg q - \frac{F(f,g,h)(t) - G(f,g,h)(t)}{H(f,g,h)(t) + 1}} \\
                    \overset{def\ a, b, c}{<} \abs{\frac{1}{\nu^2 + 1}(\xi - q)} + \frac{1}{2t+2}                         \\
                    \overset{\forall \nu:\; \frac{1}{\nu^2 + 1} \leq 1}{\leq} |\xi - q| + \frac{1}{2t+2}                  \\
                    \overset{def\ q \Rightarrow |\xi - q| < \frac{1}{2t+2}}{<} \frac{1}{2t + 2} + \frac{1}{2t+2} = \frac{1}{t+1}
                \end{split}
            \end{equation}
            Значи $\langle F(f, g, h), G(f, g, h), H(f, g, h)\rangle$ е име на $\arctan \xi$
    \end{itemize}
\end{solution}

\begin{problem}
Дайте пример за реална функция $\theta : [0, 1] \to  \R$, такава че $\theta(0) = 0$, ограничението на $\theta$ до $(0, 1]$ е изчислимо, $\theta$ е непрекъсната в 0, но не е ефективно непрекъсната в 0.
\end{problem}
\begin{solution}
    Ако допуснем, че е ефектнивно непрекъсната в 0 - съществува $r : \N \to \N$ - изчислима:
    \begin{equation}
        \forall \xi \in [0; 1],\ t \in \N:\; \xi < \frac{1}{r(t) + 1} \Rightarrow \abs{\theta(\xi)} < \frac{1}{t+1}
    \end{equation}
    \begin{equation}
        \theta(x) = \frac{\sin x}{x} - 1
    \end{equation}
    Нека имаме $\xi \in [0; 1]$ и за някакво $t$ е вярно:
    \begin{equation}
        \xi < \frac{1}{r(t) + 1}
    \end{equation}
    Тогава:
    \begin{equation}
        \theta(\xi) = \frac{sin \xi}{\xi} - 1 > \frac{\sin \xi }{r(t) + 1} - 1 > \frac{-1}{r(t) + 1} - 1 = \frac{-1 + r(t) + 1}{r(t) + 1} = \frac{r(t)}{r(t) + 1} \overset{?}{\sim} 1
    \end{equation}
\end{solution}