\begin{problem}[Ефективни Болцано-Вайерщрас]
    Докажете, че всяка ограничена редица от реални числа има ефективно сходяща подредица.
\end{problem}
\begin{solution}
    Нека $\{a_n \in \R\}_{n \in \N}$ е редица от реални числа.

    Щом е ограничена, значи има реални числа $m, M \in \R$, т.ч.:
    \begin{equation}
        \forall n\in \N:\; m \leq a_n \leq M
    \end{equation}
    Подобно на класическото доказателство, ще направим "нормализация" \ :
    \begin{equation}
        \left\{b_n = \frac{a_n - m}{M-m}\right\}_{n \in \N}
    \end{equation}
    Сега имаме, че $\forall n\in\N:\; b_n \in [0;1]$. Ясно е, че дефинираната трансформация е обратима с обратна функция $a_n = (M-m)(b_n+m)$.

    И ще формулираме следните 2 редици:
    \begin{equation}
        \left\{c^{(1l)}_n = b_n \mid b_n < \frac{1}{2}\right\},\  \left\{c^{(1r)}_n = b_n \mid b_n \geq \frac{1}{2}\right\}
    \end{equation}
    Гледаме в коя редица има безкраен брой елементи - нея определяме за "главна" \ в следващата стъпка на процедурата\footnote{не е \emph{алгоритъм} понеже никога не завършва - разпространен термин е \emph{изчислителен процес}. В случая дори не можем да твърдим, че се случва изчисление понеже избора на главна редица се базира на решение дали дадена редица е безкрайна}, а другата ще наричаме "вторична" . В случай, че и в двете има, за определеност вземаме за "главна" \ тази, чийто горен индекс съдържа $l$.

    От вторичната редица, избираме произволен елемент. Ако е крайна, избираме най-малката, ако разглеждаме $r$ редица или най-големия, ако разглеждаме $l$ редица. Ако вторичната редица се окаже безкрайна - избираме елемента с най-малък индекс. Означаваме този елемент с $c_1$.

    Главната редица остава ограничена - или от $\left[0;\frac{1}{2}\right)$, или от $\left[\frac{1}{2};1\right]$. Нека границите са $[i_2; j_2]$. Тогава разглеждаме редиците:
    \begin{equation}
        \left\{c^{(2l)}_n = b_n \mid b_n < \frac{j_2-i_2}{2}\right\},\  \left\{c^{(2r)}_n = b_n \mid b_n \geq \frac{j_2-i_2}{2}\right\}
    \end{equation}

    Отново избираме главна и вторична редица по горните критерии, както и елемент $c_2$ аналогично.

    Изобщо, на повторение $n \in \N$ от процедурата имаме, че главната редица е ограничена в интервал $[i_n; j_n]$ (отчитаме, че дължината на интервала е $\frac{1}{2^n}$). Дефинираме:
    \begin{equation}
        \left\{c^{((n+1)l)}_n = b_n \mid b_n < \frac{b_n-a_n}{2}\right\},\  \left\{c^{((n+1)r)}_n = b_n \mid b_n \geq \frac{j_n-i_n}{2}\right\}
    \end{equation}
    Отново избираме главна, вторична и елемент $c_{n+1}$ по аналогичен начин както до сега.

    Отчитаме, че точки на сгъстяване (граници на подредици), могат да съществуват само в околности, където има безкраен брой точки.
    
    Сега да разгледаме редицата $\{c_n\}_{n\in\N}$ - очевидно е подредица на $\{b_n\}_{n\in\N}$. От класическото доказателство знаем, че е сходяща към някоя точка на сходимост на $\{b_n\}_{n\in\N}$. Нещо повече - редицата $\{c_n\}_{n\in\N}$ изпълнява свойството на Коши за фундаменталност:
    \begin{equation}\label{eq:bol-wei-fundamental}
        \forall n,\ m\in \N:\; |c_n - c_{n+m}| \leq \frac{1}{2^{(n-1)}}
    \end{equation}

    В затворения интервал $[0;1] \subset \R$, фундаменталната редица има граница вътре в интервала, затова нека:
    \begin{equation}
        \lim\limits_{n\to\infty} c_n = C \in [0; 1]
    \end{equation}
    Сега можем да разгледаме изискването редицата ефективно да схожда към $C$ - търсим изчислима ф-ия $\delta :\N \to \N$, т.ч.:
    \begin{equation}
        \forall n>\delta(k):\; \left|c_n - C\right| < \frac{1}{k+1}
    \end{equation}

    От \eqref{eq:bol-wei-fundamental} видяхме, че елементите стават произволно близо. Значи можем да оценим по подобен начин разстоянието до границата $C$, понеже оценката на разстоянието не зависи от индекса на второто число:
    \begin{equation}
        \forall n\in\N^+:\; \left|c_n - \lim\limits_{m\to\infty} c_{n+m}\right| \leq \frac{1}{2^{(n-1)}} < \frac{1}{n-1}
    \end{equation}
    Нека дефинираме:
    \begin{equation}
        \delta(k) = k + 2
    \end{equation}
    Така при $n > \delta(k)$, имаме:
    \begin{equation}
        \left|c_n - \lim\limits_{m\to\infty} c_{n+m}\right| \leq \frac{1}{2^{(n-1)}} < \frac{1}{2^{(k+1)}} \leq \frac{1}{k+1}
    \end{equation}
    Значи наистина $\delta$ е регулатор на сходимост.
\end{solution}